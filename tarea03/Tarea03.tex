\documentclass[a4paper,12pt]{article}
\usepackage[utf8]{inputenc}
\usepackage[spanish]{babel}
\usepackage{amsmath}
\usepackage{graphicx}
\usepackage{longtable}
\usepackage{amsfonts}
\usepackage{amssymb}
\usepackage[a4paper,margin=1in]{geometry}
\usepackage{fancyhdr}
\usepackage{titlesec}
\usepackage{titling}
\usepackage{xcolor}
\usepackage{booktabs}
\usepackage{float}

\usepackage{xcolor} % Paquete para colores
\usepackage{enumitem} % Paquete para personalizar listas

\usepackage[breaklinks=true]{hyperref}  % Permite el ajuste de los enlaces largos
\usepackage{url}  % Paquete para manejar URLs largas
\usepackage{breakurl}  % Paquete adicional para romper URLs
\usepackage{microtype}  % Mejora el ajuste del texto
\usepackage{microtype}
\def\UrlBreaks{\do\/\do-\do_\do.\do:\do\?}
\usepackage[backend=biber]{biblatex} % Paquete para gestionar las referencias\addbibresource{referencias.bib} 
\addbibresource{referencias.bib} % Añadir el archivo .bib


\begin{document}

\begin{titlepage}
    \centering
    \vspace*{1cm}
    \textbf{UNIVERSIDAD NACIONAL AUTÓNOMA DE MÉXICO}
    
    \vspace{0.5cm}
    \textbf{FACULTAD DE CIENCIAS, 2025-II}
    
    \vspace{0.5cm}
    \textbf{Organización y Arquitectura de Computadoras}
    
    \vspace{5 cm}
    \textbf{\LARGE \textcolor{teal}{TAREA 03:}}
    \vspace{1cm}\\
    \textbf{\LARGE \textcolor{teal}{Lógica digital}}
    \vspace{3cm}\\

    \Large Baños Mancilla Ilse Andrea - 321173988\\
    \vspace{1 cm}
    \Large Rivera Machuca Gabriel Eduardo 321057608
    
  
    \vspace{1cm}
\end{titlepage}

\section*{\textcolor{teal}{Preguntas}}

\begin{enumerate}[label=\textcolor{teal}{\textbf{\arabic*.}}]
%-------------Pregunta 1-------------------------------------------
    \item Demuestra que $x(yx)=(xy)z$ 

            
      
%-------------Pregunta 2-------------------------------------------
    \item Demuestra si la siguiente igualdad es válida $x(\overline{x}+y)=xy$ 
        
            


%-------------Pregunta 3-------------------------------------------
    \item  Demuestra si la siguiente igualdad es válida $(x+y)(\overline{x}+z)(y+z)=(x+y)(\overline{x}+z)$
        
            

%-------------Pregunta 4-------------------------------------------            
    \item  Demuestra si la siguiente igualdad es válida $\overline{xy}=\overline{x} \cdot \overline{y}$
        
        
%-------------Pregunta 5-------------------------------------------

    \item Verifica la siguiente igualdad usando los postulados de Huntington
            \begin{center}
                $F(x,y,z) = x + x(\overline{x} + y) + \overline{x} y = x + y$
            \end{center}
           
%-------------Pregunta 6-------------------------------------------

    \item Obten los mintérminos y reduce la siguiente función 
        \begin{center}
            $F(x,y,z) = \overline{x} \cdot \overline{y} \cdot \overline{z} \cdot x + \overline{z} \cdot x + z \cdot x + x \cdot \overline{y} + \overline{z} $
        \end{center}
    
    
%-------------Pregunta 7-------------------------------------------

    \item Simplifica la siguiente función usando su tabla de verdad asociada y mapas de Karnaugh.
    
        \begin{center}
            $F(x,y,z) = \overline{xyz} + \overline{xy} z + \overline{x} y \overline{z} + x \overline{yz} +  \overline{x} yz +x \overline{y} z +xyz $
        \end{center}


%-------------Pregunta 8-------------------------------------------
    \item Reduce la siguiente función y da sus maxitérminos
        \begin{center}
            $F(x,y,z) = (x + \overline{x} z) \cdot (\overline{y} + \overline{z}) z$
        \end{center}

%-------------Pregunta 9-------------------------------------------
    \item Utilizando Mapas de Karnaugh simplifica la función.
    
        \begin{center}
            $F(x_0,x_1,x_2,x_3) =
            \overline{x_0 x_1 x_2 x_3} 
            +\overline{x_0 x_1 x_2} x_3
            +\overline{x_0 x_1} x_2 x_3
            +x_0 \overline{x_1} x_2 x_3
            + x_0 x_1 \overline{x_2 x_3}
            + \overline{x_0} x_1 \overline{x_2 x_3}
            x_0 x_1 x_2 x_3
            $
        \end{center}
        
    
%-------------Pregunta 10-------------------------------------------
    \item Para realizar una Mapa de Karnaugh con más de 5 variables se mencionó que existe más de
    una forma de representarlo.\\
    Investiga ambos métodos y utiliza el que más se te acomode para reducir la siguiente función.

    \begin{center}
        $F(x_0,x_1,x_2,x_3,x_4) = 
        \overline{x_0 x_1 x_2 x_3 x_4} 
        + \overline{x_0 x_1 x_2} x_3 \overline{x_4} 
        + \overline{x_0 x_1} x_2 x_3 \overline{x_4} 
        + x_0 \overline{x_1} x_2 x_3 x_4
        + x_0 x_1 \overline{x_2 x_3} x_4
        + \overline{x_0} x_1 \overline{x_2 x_3} x_4
        + x_0 x_1 x_2 x_3 x_4 
        $
    \end{center}
   
%-------------Pregunta 11-------------------------------------------
    \item Utilizando el algoritmo de Quine-McCluskey realiza la siguiente reducción.

        \begin{center}
            $F(x_0,x_1,x_2,x_3,x_4) = 
            \overline{x_0 x_1 x_2 x_3 x_4} 
            + \overline{x_0 x_1 x_2} x_3 \overline{x_4} 
            + \overline{x_0} x_1 x_2 x_3 \overline{x_4}   
            + \overline{x_0} x_1 \overline{x_2} x_3 \overline{x_4}  
            + x_0 \overline{x_1} x_2 x_3 x_4    
            + x_0 x_1 \overline{ x_2 x_3} x_4 
            + \overline{x_0} x_1 \overline{ x_2 x_3} x_4 
            + x_0 x_1 x_2 \overline{ x_3} x_4 
            + x_0 x_1 x_2 x_3 x_4 
             $
        \end{center}

%-------------Pregunta 12-------------------------------------------
    \item Utilizando el algoritmo de Quine-McCluskey realiza la siguiente reducción.
        \begin{center}
            $F(x_0,x_1,x_2,x_3,x_4) = 
            \overline{x_0 x_1 x_2 x_3 x_4} 
            + \overline{x_0 x_1 x_2} x_3 \overline{x_4} 
            + \overline{x_0 x_1} x_2 x_3 \overline{x_4}  
            + \overline{x_0 x_1} x_2 x_3 x_4  
            + \overline{x_0} x_1 x_2 x_3 \overline{x_4} 
            + \overline{x_0} x_1 \overline{x_2} x_3 \overline{x_4} 
            + x_0 \overline{x_1} x_2 x_3 x_4
            + x_0 x_1 \overline{x_2 x_3} x_4
            + \overline{x_0} x_1 \overline{x_2 x_3} x_4
            + x_0 x_1 x_2 \overline{ x_3} x_4 
            + x_0 x_1 x_2 x_3 x_4 
            $
        \end{center}
    


\end{enumerate}


%--\nocite{*}
%--\printbibliography

\end{document}

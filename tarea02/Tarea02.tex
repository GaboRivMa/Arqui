\documentclass[a4paper,12pt]{article}
\usepackage[utf8]{inputenc}
\usepackage[spanish]{babel}
\usepackage{amsmath}
\usepackage{graphicx}
\usepackage{longtable}
\usepackage{amsfonts}
\usepackage{amssymb}
\usepackage[a4paper,margin=1in]{geometry}
\usepackage{fancyhdr}
\usepackage{titlesec}
\usepackage{titling}
\usepackage{xcolor}
\usepackage{booktabs}
\usepackage{float}

\usepackage[backend=biber]{biblatex} % Paquete para gestionar las referencias\addbibresource{referencias.bib} 
\addbibresource{referencias.bib} % Añadir el archivo .bib


\begin{document}

\begin{titlepage}
    \centering
    \vspace*{1cm}
    \textbf{UNIVERSIDAD NACIONAL AUTÓNOMA DE MÉXICO}
    
    \vspace{0.5cm}
    \textbf{FACULTAD DE CIENCIAS, 2025-II}
    
    \vspace{0.5cm}
    \textbf{Organización y Arquitectura de Computadoras}
    
    \vspace{5 cm}
    \textbf{\LARGE \textcolor{teal}{TAREA 02:}}
    \vspace{3cm}\\

    \Large Baños Mancilla Ilse Andrea - 321173988\\
    \vspace{1 cm}
    \Large Gabriel Eduardo Rivera Machuca 321057608
    
  
    \vspace{1cm}
\end{titlepage}

\section*{\textcolor{teal}{Preguntas}}

\begin{enumerate}
        \item La Arquitectura de Computadoras se dedica  únicamente al estudio de las instrucciones de una computadora y su desempeño respecto a estas ¿sí, no? Argumenta tu respuesta.
        \item ¿Los registros son dispositivos de hardware que permiten almacenar cualquier valor en binario? Argumenta tu respuesta.
        \item ¿Cuál es la diferencia entre un AMD Ryzen 5 y un Intel Core i5? ¿Qué tipo de organización de computadoras o microarquitectura tiene?
        \item De los dos tipos de arquitecturas, RISC y CISC. ¿Cuál de las dos requiere un mayor número de instrucciones para realizar una tarea? ¿Por qué crees que así sea?
        \item Menciona los tres factores de desempeño y de que dependen cada uno.
        \item Un programa tarda 9 millones de ciclos en una computadora cuyo ciclo dura 3 ns. ¿Cuál es el tiempo de CPU?
        \item Un programa tarda 14 millones de ciclos en una máquina a 2.4 GHz. ¿Cuál es el tiempo de CPU?
        \item ¿En una arquitectura CISC el periodo de una señal de reloj puede ser más grande que en una  arquitectura RISC?
        \item El Intel 4004 (i4004), un CPU de 4 bits, fue el primer microprocesador en un simple chip, así como el primero disponible comercialmente y contenía 2300 transistores. Utilizando la Ley de Moore, ¿cuántos transistores se esperaría que tuviera hoy en día?     
        \item El Intel Core i9-9900K es un procesador de 64 bits con 8 núcleos con tecnología Hyper-Threading de Intel, la cual ejecuta 2 hilos en cada núcleo por lo que cuenta con 16 hilos de procesamiento en total. El Intel Core i9-9900K cuenta con 3052 mil millones de transistores.Comparando con tu respuesta anterior ¿Es mayor o menor a lo esperado? ¿Se cumplió la ley de Moore? Argumenta tu respuesta.



            
    
\end{enumerate}


\nocite{*}
\printbibliography

\end{document}
